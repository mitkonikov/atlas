\subsection{Basic Algorithms}

\subsubsection{Binary Exponentiation}

\subsubsection{Sum of Geometric Progression}

Given integers $A$, $N$ and $M$, find $\displaystyle \sum_{i=0}^{N-1} A^i$, mod $M$.

\begin{lstlisting}
int geometric(int N, int A, int M) {
  ll T = 1;
  ll E = A % M;
  ll total = 0;
  while (N > 0) {
    if (N & 1) {
      total = (E * total + T) % M;
    }
    T = ((E + 1) * T) % M;
    E = (E * E) % M;
    N = N / 2;
  }
  return total;
}
\end{lstlisting}

\subsection{Primes}

\subsubsection{Carmichael Lambda (Universal Totient Function)}

$\lambda(n)$ is a smallest number that satisfies $a^{\lambda(n)} \equiv 1$ (mod $n$)
for all $a$ coprime with $n$. This is also known as an universal totien tunction $\psi(n)$.

Complexity:
\lstset{basicstyle={\small\ttfamily}\ttfamily,style=smaller_code}
\begin{itemize}
  \item \lstinline{carmichael_lambda(n)} ... $O(\sqrt{n})$ by trial division.
  \item \lstinline{carmichael_lambda(lo,hi)} ... $O((H-L) loglog(H))$ by prime sieve.
\end{itemize}
\lstset{basicstyle=\fontsize{7}{9}\ttfamily,style=smaller_code}

\textit{Note.} Required GCD and LCM.

\begin{lstlisting}
ll carmichael_lambda(ll n) {
  ll lambda = 1;
  if (n % 8 == 0) n /= 2;
  for (ll d = 2; d*d <= n; ++d) {
    if (n % d == 0) {
      n /= d;
      ll y = d - 1;
      while (n % d == 0) {
        n /= d;
        y *= d;
      }
      lambda = lcm(lambda, y);
    }
  }
  if (n > 1) lambda = lcm(lambda, n-1);
  return lambda;
}
\end{lstlisting}
\newpage
\begin{lstlisting}
  vector<ll> primes(ll lo, ll hi) { // primes in [lo, hi)
  const ll M = 1 << 14, SQR = 1 << 16;
  vector<bool> composite(M), small_composite(SQR);

  vector<pair<ll,ll>> sieve; 
  for (ll i = 3; i < SQR; i+=2) {
    if (!small_composite[i]) {
      ll k = i*i + 2*i*max(0.0, ceil((lo - i*i)/(2.0*i)));
      sieve.push_back({2*i, k});
      for (ll j = i*i; j < SQR; j += 2*i) 
        small_composite[j] = 1;
    }
  }
  vector<ll> ps; 
  if (lo <= 2) { ps.push_back(2); lo = 3; }
  for (ll k = lo|1, low = lo; low < hi; low += M) {
    ll high = min(low + M, hi);
    fill(all(composite), 0);
    for (auto &z: sieve) 
      for (; z.snd < high; z.snd += z.fst)
        composite[z.snd - low] = 1;
    for (; k < high; k+=2) 
      if (!composite[k - low]) ps.push_back(k);
  }
  return ps;
}
vector<ll> primes(ll n) { // primes in [0,n)
  return primes(0,n);
}
vector<ll> carmichael_lambda(ll lo, ll hi) { // lambda(n) for all n in [lo, hi)
  vector<ll> ps = primes(sqrt(hi)+1);
  vector<ll> res(hi-lo), lambda(hi-lo, 1);
  iota(all(res), lo);

  for (ll k = ((lo+7)/8)*8; k < hi; k += 8) res[k-lo] /= 2;
  for (ll p: ps) {
    for (ll k = ((lo+(p-1))/p)*p; k < hi; k += p) {
      if (res[k-lo] < p) continue;
      ll t = p - 1;
      res[k-lo] /= p;
      while (res[k-lo] > 1 && res[k-lo] % p == 0) {
        t *= p;
        res[k-lo] /= p; 
      }
      lambda[k-lo] = lcm(lambda[k-lo], t);
    }
  }
  for (ll k = lo; k < hi; ++k) {
    if (res[k-lo] > 1) 
      lambda[k-lo] = lcm(lambda[k-lo], res[k-lo]-1);
  }
  return lambda; // lambda[k-lo] = lambda(k)
}
\end{lstlisting}

\newpage
\subsection{Modular Arithmetic}

\begin{center}
\begin{minipage}[t]{0.45\linewidth}
\subsubsection{Discrete Log}

The following function returns any $x$ such that $a^x \equiv b$ $(mod$ $m)$ in $O(\sqrt{m})$ time.
It returns $-1$ if such $x$ can not be found. \cite{DiscreteLogCPAlgo}
  
\begin{lstlisting}
int solve(int a, int b, int m) {
  // This reduction step is not needed
  // If a and m are relatively prime.
  a %= m, b %= m;
  int k = 1, add = 0, g;
  while ((g = gcd(a, m)) > 1) {
    if (b == k)
      return add;
    if (b % g)
      return -1;
    b /= g, m /= g, ++add;
    k = (k * 1ll * a / g) % m;
  }

  // Here the algorithm starts
  int n = sqrt(m) + 1;
  int an = 1;
  for (int i = 0; i < n; ++i)
    an = (an * 1ll * a) % m;

  unordered_map<int, int> vals;
  for (int q = 0, cur = b; q <= n; ++q) {
    vals[cur] = q;
    cur = (cur * 1ll * a) % m;
  }

  // if the gcd(a, m) = 1
  //   => cur = 1 AND add = 0
  for (int p = 1, cur = k; p <= n; ++p) {
    cur = (cur * 1ll * an) % m;
    if (vals.count(cur)) {
      int ans = n * p - vals[cur] + add;
      return ans;
    }
  }
  return -1;
}
\end{lstlisting}
\end{minipage}
\qquad
\begin{minipage}[t]{0.45\linewidth}
\subsubsection{Discrete Root}

The problem of finding a discrete root is defined as follows.
Given a prime $n$ and two integers and $a$, $k$, find all $x$ for which:
$x^k \equiv a$ $(mod$ $n)$ \cite{DiscreteRootCPAlgo}

Required functions:
{\setstretch{0.5}
% \lstset{basicstyle={\small\ttfamily}\ttfamily,style=smaller_code}
\begin{itemize}
  \item{\lstinline{powmod(a, x, m)} ... $a^x$ $(mod$ $m)$ in $O(log(n))$}
  \item{\lstinline{gcd(a, b)} ... GCD of $a$ and $b$}
  \item{\lstinline{primitive_root(m)} ... The primitive root of $m$}
\end{itemize}
% \lstset{basicstyle=\fontsize{7}{9}\ttfamily,style=smaller_code}
}

\begin{lstlisting}
// Find all integers x such that x^k = a (mod n)
void solve(int n, int k, int a) {
  if (a == 0) {
    puts("1\n0");
    return;
  }

  int g = primitive_root(n);

  // Baby-step giant-step discrete logarithm algorithm
  int sq = (int) sqrt (n + .0) + 1;
  vector<pair<int, int>> dec(sq);
  for (int i = 1; i <= sq; ++i)
    dec[i-1] = {powmod(g, i * sq * k % (n - 1), n), i};
  sort(dec.begin(), dec.end());
  int any_ans = -1;
  for (int i = 0; i < sq; ++i) {
    int my = powmod(g, i * k % (n - 1), n) * a % n;
    auto it = lower_bound(dec.begin(), dec.end(), make_pair(my, 0));
    if (it != dec.end() && it->first == my) {
      any_ans = it->second * sq - i;
      break;
    }
  }
  if (any_ans == -1) {
    puts("0");
    return;
  }

  // Print all possible answers
  int delta = (n-1) / gcd(k, n-1);
  vector<int> ans;
  for (int cur = any_ans % delta; cur < n-1; cur += delta)
    ans.push_back(powmod(g, cur, n));
  sort(ans.begin(), ans.end());
  printf("%d\n", ans.size());
  for (int answer : ans)
    printf("%d ", answer);
}

\end{lstlisting}
\end{minipage}
\end{center}

\newpage

\begin{center}
\begin{minipage}[t]{0.45\linewidth}
\subsubsection{Primitive Root}

\begin{lstlisting}
// Finds the primitive root modulo p
int primitive_root(int p) {
  vector<int> fact;
  int phi = p-1, n = phi;
  for (int i = 2; i * i <= n; ++i) {
    if (n % i == 0) {
      fact.push_back(i);
      while (n % i == 0)
        n /= i;
    }
  }
  if (n > 1)
    fact.push_back(n);

  for (int res = 2; res <= p; ++res) {
    bool ok = true;
    for (int factor : fact) {
      if (powmod(res, phi / factor, p) == 1) {
        ok = false;
        break;
      }
    }
    if (ok) return res;
  }
  return -1;
}
\end{lstlisting}
\end{minipage}
\qquad
\begin{minipage}[t]{0.45\linewidth}
\subsubsection{Factorial Mod Linear P}

\begin{lstlisting}
int factmod(int n, int p) {
  // Precompute in O(p)
  vector<int> f(p);
  f[0] = 1;
  for (int i = 1; i < p; i++)
    f[i] = f[i-1] * i % p;

  // Answer in O(log(n))
  int res = 1;
  while (n > 1) {
    if ((n/p) % 2)
      res = p - res;
    res = res * f[n%p] % p;
    n /= p;
  }
  return res;
}
\end{lstlisting}

\subsubsection{Legendre's Formula}

The exponent of $p$ in the prime factorization \cite{FactorialModP_CPAlgo} of $n!$ is:

\begin{center}
  $\displaystyle \nu_p(n!) = \sum_{i=1}^{\infty} \left\lfloor \frac{n}{p^i} \right\rfloor$
\end{center}

\begin{lstlisting}
int multiplicity_factorial(int n, int p) {
  int count = 0;
  do { n /= p; count += n; } while (n);
  return count;
}
\end{lstlisting}
\end{minipage}
\end{center}

\newpage

\subsubsection{ModInt Structure}

\begin{lstlisting}
const int mod=998244353;
struct mi {
    int v;
    mi(){v=0;}
    mi(ll _v){v=int(-mod<=_v&&_v<mod?_v:_v%mod); if(v<0)v+=mod;}
    explicit operator int()const{return v;}
    friend bool operator==(const mi &a,const mi &b){return (a.v==b.v);}
    friend bool operator!=(const mi &a,const mi &b){return (a.v!=b.v);}
    friend bool operator<(const mi &a,const mi &b){return (a.v<b.v);}
    mi& operator+=(const mi &m){if((v+=m.v)>=mod)v-=mod; return *this;}
    mi& operator-=(const mi &m){if((v-=m.v)<0)v+=mod; return *this;}
    mi& operator*=(const mi &m){v=((ll)(v)*m.v)%mod; return *this;}
    mi& operator/=(const mi &m){return (*this)*=inv(m);}
    friend mi pow(mi a,ll e){mi r=1; for(;e;a*=a,e/=2)if(e&1)r*=a; return r;}
    friend mi inv(mi a){return pow(a,mod-2);}
    mi operator-()const{return mi(-v);}
    mi& operator++(){return (*this)+=1;}
    mi& operator--(){return (*this)-=1;}
    friend mi operator++(mi &a,int){mi t=a; ++a; return t;}
    friend mi operator--(mi &a,int){mi t=a; --a; return t;}
    friend mi operator+(mi a,const mi &b){return a+=b;}
    friend mi operator-(mi a,const mi &b){return a-=b;}
    friend mi operator*(mi a,const mi &b){return a*=b;}
    friend mi operator/(mi a,const mi &b){return a/=b;}
    friend istream& operator>>(istream &is,mi &m){ll _v; is >> _v; m=mi(_v); return is;}
    friend ostream& operator<<(ostream &os,const mi &m){os << m.v; return os;}
};
\end{lstlisting}

\subsection{Binomial Coefficient}

Precomputing binomial coefficients in $O(N^2)$.
\begin{lstlisting}
const int maxn = ...;
int C[maxn + 1][maxn + 1];
C[0][0] = 1;
for (int n = 1; n <= maxn; ++n) {
    C[n][0] = C[n][n] = 1;
    for (int k = 1; k < n; ++k)
        C[n][k] = C[n - 1][k - 1] + C[n - 1][k];
}
\end{lstlisting}

Computing binomial coefficients in $O(\log M)$, where $M$ is the modulo.
\begin{lstlisting}
mi C(mi n, mi k) {
  return fact[n] / (fact[k] * fact[n-k]);
}
\end{lstlisting}


\newpage
\subsection{Inclusion-Exclusion Principle}

A general example code to showcase the inclusion-exclusion principle.

\begin{lstlisting}
for (int i = 1; i < (1 << n); i++) {
  vector<int> current;
  int bit_count = 0;
  for (int bit = 0; bit < n; bit++) {
    if ((1 << bit) & i) {
      // add the i-th element to
      // the data structure
      add(current, element[i])
      bit_count++;
    }
  }

  // depending on the parity of the bitcount,
  // calculate and modify the answer
  if (bit_count % 2 == 0) {
    ans -= calc(current);
  } else {
    ans += calc(current);
  }
}

\end{lstlisting}

\subsection{Dynamic Programming - Sum over subsets}

\begin{center}
$\displaystyle F_{mask} = \sum_{i \subseteq mask}^{} A_i$
\end{center}

\textbf{Time Complexity: $O(N \cdot 2^N)$}

\begin{lstlisting}
for (int i = 0; i < (1 << N); ++i) F[i] = A[i];
for (int i = 0; i < N; ++i) {
  for (int mask = 0; mask < (1 << N); ++mask) {
  	if(mask & (1<<i)) F[mask] += F[mask^(1<<i)];
  }
}
\end{lstlisting}

If we want for each subset to sum up the functions with inclusion-exclusion in mind, we can use the Mobius Transform.

\begin{center}
  $\displaystyle \mu(f(s)) = \sum_{s' \subseteq s} (-1)^{|s \setminus s'|} f(s')$
\end{center}

\textbf{Time Complexity: $O(N \cdot 2^N)$}

\begin{lstlisting}
for (int i = 0; i < N; i++) {
  for (int mask = 0; mask < (1 << N); mask++) {
    if ((mask & (1 << i)) != 0) {
      f[mask] -= f[mask ^ (1 << i)];
    }
  }
}
for(int mask = 0; mask < (1 << N); mask++) {
  zinv[mask] = mu[mask] = f[mask];
}
\end{lstlisting}

\newpage

\subsection{Fractional Binary Search}
\subsection{Miller Rabin Primality Test}
\subsection{Matrix Exponentiation}
\subsection{Phi function}
\subsection{Gaussian Elimination}
\subsection{Determinant}
\subsection{K-th Permutation}
\subsection{Berlekamp-Messay Algorithm}
\subsection{Chinese Remainder Theorem}

\hrulefill
\begin{multicols}{2}
\inputminted[autogobble,fontsize=\footnotesize]{C++}{Maths/crt.cpp}
\end{multicols}
\hrulefill

\subsection{Mobius Inversion}
\subsection{GCD \& LCM Convolution}
\subsection{Fast Fourier Transform} 
\subsection{Number Theoretic Transform}
\subsection{Fast Subset Transform}
\subsection{More Operations on Finite Polynomials}
\subsection{Game Theory, Nim Game}
\subsection{Simplex}

\newpage

\subsection{Miscellaneous Stuff}

\subsubsection{Gray Code}

Gray code is a binary numeral system where two successive values differ in only one bit. \cite{GrayCode_CPAlgo}
For example, the sequence of Gray codes for 3-bit numbers is: 000, 001, 011, 010, 110, 111, 101, 100...

\begin{lstlisting}
int g(int n) {
  return n ^ (n >> 1);
}

int rev_g(int g) {
  int n = 0;
  for (; g; g >>= 1)
    n ^= g;
  return n;
}
\end{lstlisting}

\subsubsection{Lemmas}

\begin{lemma}
$1 + 3 + 5 + \cdots + (2n - 1) = n^2$
\end{lemma}

\begin{lemma}
$\displaystyle \sum_{i=1}^{n} i \cdot i! = (n + 1)! - 1$
\end{lemma}

\begin{lemma}
  $2^n < n!$ , $n > 3$
\end{lemma}

\begin{lemma}
  $|a_1+a_2+\cdots+a_n| \leq |a_1| + |a_2| + \cdots + |a_n|$, for any real numbers $a_1, a_2, \dots a_n$
\end{lemma}

\begin{lemma}
  For any array $(a_1, a_2, \dots a_n)$, there exist $l$ and $r$, such that $1 \leq l < r \leq n$ and \\ $\displaystyle \sum_{i=l}^{r} a_i \equiv 0$ (mod n).
\end{lemma}
