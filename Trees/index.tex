\subsection{Least Common Ancestor}

\hrulefill \vspace{-\baselineskip}
\begin{multicols}{2}
\inputminted[autogobble,fontsize=\footnotesize]{C++}{Trees/lca.cpp}
\end{multicols}
\vspace{-\baselineskip}
\noindent \hrulefill

\newpage

\subsection{Heavy-Light Decomposition}

\hrulefill \vspace{-\baselineskip}
\begin{multicols}{2}
\inputminted[autogobble,fontsize=\footnotesize]{C++}{Trees/hld.cpp}
\end{multicols}
\vspace{-\baselineskip}
\noindent \hrulefill

\subsection{Centroid Decomposition}

\hrulefill \vspace{-\baselineskip}
\begin{multicols}{2}
\inputminted[autogobble,fontsize=\footnotesize]{C++}{Trees/centroid.cpp}
\end{multicols}
\vspace{-\baselineskip}
\noindent \hrulefill

\subsection{Kruskal Reconstruction Trees}
\subsection{Minimum Diameter Spanning Tree}

\hrulefill \vspace{-\baselineskip}
\begin{multicols}{2}
\inputminted[autogobble,fontsize=\tiny]{C++}{Trees/mdst.cpp}
\end{multicols}
\vspace{-\baselineskip}
\noindent \hrulefill

\subsection{Gomory Hu Tree}

\newpage

\subsection{Dominator Tree (Lengauer-Tarjan)}

Let $G = (V, E)$ be a directed graph and fix $r$ in $V$.
$v$ is a dominator of $u$ if all paths from $r$ to $u$ go through $v$.
The set of dominators of $u$ forms a total order, and 
the closest dominator is called the immediate dominator.
The set $\{ (u,v) : v$ is the immediate dominator $\}$ forms a tree, 
which is called the dominator tree. \cite{TarjanLengauer1979Jan}

% https://www.spoj.com/problems/BIA/
% https://github.com/mukel/ProgrammingContests/blob/master/OldStuff/SPOJ/new4/bia.cpp

\textbf{Time Complexity: $O(M log N)$}\\
\begin{lstlisting}
struct edge { int src, dst; };
struct graph {
  int n;
  vector<vector<edge>> adj, rdj;
  graph(int n = 0) : n(0) { } 
  void add_edge(int src, int dst) {
    n = max(n, max(src, dst)+1);
    adj.resize(n); rdj.resize(n);
    adj[src].push_back({src, dst});
    rdj[dst].push_back({dst, src});
  }
  vector<int> rank, semi, low, anc;
  int eval(int v) { 
    if (anc[v] < n && anc[anc[v]] < n) {
      int x = eval(anc[v]);
      if (rank[semi[low[v]]] > rank[semi[x]]) low[v] = x;
      anc[v] = anc[anc[v]];
    }
    return low[v];
  }
  vector<int> prev, ord;
  void dfs(int u) {
    rank[u] = ord.size();
    ord.push_back(u);
    for (auto e: adj[u]) {
      if (rank[e.dst] < n) continue;
      dfs(e.dst);
      prev[e.dst] = u;
    }
  }
  vector<int> idom; // idom[u] is an immediate dominator of u
  void dominator_tree(int r) {
    idom.assign(n, n); prev = rank = anc = idom;
    semi.resize(n); iota(all(semi), 0); low = semi;
    ord.clear(); dfs(r);

    vector<vector<int>> dom(n);
    for (int i = ord.size()-1; i >= 1; --i) {
      int w = ord[i];
      for (auto e: rdj[w]) {
        int u = eval(e.dst);
        if (rank[semi[w]] > rank[semi[u]]) semi[w] = semi[u];
      }
      dom[semi[w]].push_back(w);
      anc[w] = prev[w];
      for (int v: dom[prev[w]]) {
        int u = eval(v);
        idom[v] = (rank[prev[w]] > rank[semi[u]] ? u : prev[w]);
      }
      dom[prev[w]].clear();
    }
    for (int i = 1; i < ord.size(); ++i) {
      int w = ord[i];
      if (idom[w] != semi[w]) idom[w] = idom[idom[w]];
    }
  }
  vector<int> dominators(int u) {
    vector<int> S;
    for (; u < n; u = idom[u]) S.push_back(u);
    return S;
  }
};
\end{lstlisting}