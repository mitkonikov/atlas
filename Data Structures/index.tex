\subsection{Disjoint Set Union}

A Disjoint Set Union (DSU) is a data structure capable of storing sets of vertices,
merging sets and checking if two elements belong in the same set in almost $O(1)$ time.
\cite{DSUCPAlgo}
\newline

\begin{center}
\begin{minipage}[t]{0.45\linewidth}
The following implementation of a DSU includes path compression, union by size and set sizes.
\begin{lstlisting}
struct dsu {
  vector<int> parent;
  dsu(int n) : parent(n, -1) {}

  int find_set(int a) {
    if (parent[a] < 0) return a;
    return parent[a] = find_set(parent[a]);
  }

  int merge(int a, int b) {
    int x = find_set(a), y = find_set(b);
    if (x == y) return x;
    if (-parent[x] < -parent[y]) swap(x, y);
    parent[x] += parent[y];
    parent[y] = x;
    return x;
  }

  bool are_same(int a, int b) {
    return find_set(a) == find_set(b);
  }

  int size(int a) {
    return -parent[find_set(a)];
  }
};
\end{lstlisting}

In some DSU applications \cite{DSUCPAlgo}, we need to keep track of the distance from $u$
to its root. If we don't use path compression, the distance is just the number
of recursive calls. But this will be inefficient. However, we can store
the distance to the root in each node and modify the distances when doing the path compression.

\begin{lstlisting}
// the parent vector has to be 
// modified to store { v, 0 } by default
// returns { root, distance }
pair<int, int> find_set(int v) {
  if (v != parent[v].first) {
      int len = parent[v].second;
      parent[v] = find_set(parent[v].first);
      parent[v].second += len;
  }
  return parent[v];
}
\end{lstlisting}

\end{minipage}
\qquad
\begin{minipage}[t]{0.45\linewidth}
When extending the data structure to allow rollbacks,
we can't use path compression, but we will still get $O(log(n))$
time per query due to the implementation of union by size.

\subsubsection{Eval DSU}

Implementation of Evaluation-capable DSU.
\begin{lstlisting}
struct EvalDSU {
  vector<int> parent;
  vector<ll> lazy;
  function<ll(int,int)> f;

  EvalDSU(int N, function<ll(int,int)> operation) {
    parent.resize(N);
    lazy.resize(N);
    iota(parent.begin(), parent.end(), 0);
    f = operation;
  }

  // Path compression to get the O(log(N))
  int find_set(int u) {
    if (parent[u] != u) {
      int leader = find_set(parent[u]);
      lazy[u] += lazy[parent[u]];
      parent[u] = leader;
    }
    return parent[u];
  }

  bool same(int u, int v) {
    return (find_set(u) == find_set(v));
  }

  // u - is the parent
  // v - is a leader of some other set
  // Merge in O(log(N))
  void merge(int u, int v) {
    if (same(u, v)) return;
    lazy[v] += f(u, v);
    parent[v] = u;
  }
  
  // Evaluation of f(u, leader(u)) in O(log(N))
  ll eval(int u) {
    find_set(u);
    return lazy[u];
  }
};
\end{lstlisting}
\end{minipage}
\end{center}

\newpage

\subsubsection{Disjoint Set Union with Queue-like Undo}

\begin{minipage}[t]{0.45\linewidth}
\begin{lstlisting}
struct DSU {
  vector<int> rank, link;
  vector<int> stk, chkp;
  
  DSU(int n) : rank(2 * n, 0), link(2 * n, -1) {}
  
  int find(int x) {
    while (link[x] != -1) 
      x = link[x];
    return x;
  }
  void unite(int a, int b) {
    a = find(a); b = find(b);
    if (a == b) return;
    if (rank[a] > rank[b]) swap(a, b);
    stk.push_back(a);
    link[a] = b;
    rank[b] += (rank[a] == rank[b]);
  }
  
  bool Try(int a, int b) {
    if (find(2 * a + 1) == find(2 * b + 1))
      return false;
    return true;
  }
  
  void Unite(int a, int b) {
    chkp.push_back(stk.size());
    unite(2 * a, 2 * b + 1);
    unite(2 * a + 1, 2 * b);
    assert(find(2 * a) != find(2 * a + 1));
  }
  
  void Undo() {
    for (int i = chkp.back(); i < (int)stk.size(); ++i)
      link[stk[i]] = -1;
    stk.resize(chkp.back());
    chkp.pop_back();
  }
};
  
struct Upd {
  int type, a, b;
};

\end{lstlisting}
\end{minipage}
\qquad
\begin{minipage}[t]{0.45\linewidth}
\begin{lstlisting}

int main() {  
  DSU dsu(n);
  vector<Upd> upds, tmp[2];
  int rem_a = 0;
  auto pop = [&]() {
    if (rem_a == 0) {
      reverse(upds.begin(), upds.end());
      for (int i = 0; i < (int)upds.size(); ++i)
        dsu.Undo();
      for (auto& upd : upds) {
        upd.type = 0;
        dsu.Unite(upd.a, upd.b);
      }
      rem_a = upds.size();
    }
    while (upds.back().type == 1) {
      tmp[1].push_back(upds.back());
      dsu.Undo();
      upds.pop_back();
    }
    int sz = (rem_a & (-rem_a));
    for (int i = 0; i < sz; ++i) {
      assert(upds.back().type == 0);
      tmp[0].push_back(upds.back());
      dsu.Undo();
      upds.pop_back();
    }
    for (int z : {1, 0}) {
      for (; tmp[z].size(); tmp[z].pop_back()) {
        auto upd = tmp[z].back();
        dsu.Unite(upd.a, upd.b);
        upds.push_back(upd);
      }
    }
    assert(upds.back().type == 0);
    upds.pop_back();
    dsu.Undo();
    --rem_a;
  };
  auto push = [&](int a, int b) {
    upds.push_back(Upd{1, a, b});
    dsu.Unite(a, b);
  };
  
  vector<int> dp(2 * m);
  int lbound = 0;
  for (int i = 0; i < 2 * m; ++i) {
    auto [a, b] = edges[i % m];
    while (!dsu.Try(a, b)) {
      pop();
      ++lbound;
    }
    push(a, b);
    dp[i] = lbound;
  }
  
  for (int i = 0; i < q; ++i) {
    int a, b; cin >> a >> b; --a; --b;
    if (dp[a + m - 1] <= b + 1) cout << "NO\n";
    else cout << "YES\n";
  }
}
\end{lstlisting}
\end{minipage}

\subsection{Sparse Table}

Good implementation in KACTL.

\subsection{Disjoint Sparse Table}

\textbf{Time Complexity (preprocessing): $O(N log N)$}\\
\textbf{Time Complexity (query): $O(1)$}\\

\begin{lstlisting}
template <typename T>
struct DisjointSparseTable {
  vector<vector<T>> ys;
  function<T(T, T)> f;
  DisjointSparseTable(vector<T> xs, function<T(T, T)> f_) : f(f_) {
    int n = 1;
    while (n <= xs.size()) n *= 2;
    xs.resize(n);
    ys.push_back(xs);
    for (int h = 1; ; ++h) {
      int range = (2 << h), half = (range /= 2);
      if (range > n) break;
      ys.push_back(xs);
      for (int i = half; i < n; i += range) {
        for (int j = i-2; j >= i-half; --j) 
          ys[h][j] = f(ys[h][j], ys[h][j+1]);
        for (int j = i+1; j < min(n, i+half); ++j) 
          ys[h][j] = f(ys[h][j-1], ys[h][j]);
      }
    }
  }
  T prod(int i, int j) { // [i, j) query
    --j;             // __CHAR_BIT__ is usually 8
    int h = sizeof(int)*__CHAR_BIT__-1-__builtin_clz(i ^ j);
    return f(ys[h][i], ys[h][j]);
  }
};
\end{lstlisting}

\subsection{Fenwick Tree}

\hrulefill \vspace{-\baselineskip}
\begin{multicols}{2}
\inputminted[autogobble,fontsize=\footnotesize]{C++}{Data Structures/fenwick_tree.cpp}
\end{multicols}
\vspace{-\baselineskip}
\noindent \hrulefill

\newpage

\subsection{Segment Tree}

\hrulefill \vspace{-\baselineskip}
\begin{multicols}{2}
\inputminted[autogobble,fontsize=\tiny]{C++}{Data Structures/segment_tree.cpp}
\end{multicols}
\vspace{-\baselineskip}
\noindent \hrulefill

\subsection{Lazy Segment Tree}

\hrulefill \vspace{-\baselineskip}
\begin{multicols}{2}
\inputminted[autogobble,fontsize=\tiny]{C++}{Data Structures/lazy_segment_tree.cpp}
\end{multicols}
\vspace{-\baselineskip}
\noindent \hrulefill

\subsection{Sparse Lazy Segment Tree}

Good implementation in KACTL.

\subsection{Persistent Segment Tree}

\hrulefill \vspace{-\baselineskip}
\begin{multicols}{2}
\inputminted[autogobble,fontsize=\tiny]{C++}{Data Structures/persistant_segtree.cpp}
\end{multicols}
\vspace{-\baselineskip}
\noindent \hrulefill

\subsection{Persistent Fenwick Tree}

\begin{lstlisting}
struct PersistentFenwickTree {
  struct change {
    int data, id;
    change(int d = 0, int i = 0) : data(d), id(i) { }
    inline friend bool operator<(const change &a, const change &b){
      return a.id < b.id;
    }
  };

  int n, now = 0;
  vector<vector<change>> tree;

  PersistentFenwickTree(int n) {
    tree.resize(n + 100);
    this->n = n;
    for (int i = 0; i <= n; i++) {
      tree[i].push_back(change());
    }
  }
  
  void modify(int i, int x) {
    for (i++; i <= n; i += i&(-i)) {
      int new_data = max(tree[i].back().data, x);
      tree[i].push_back(change(new_data, now));
    }
    now++;
  }

  int query(int i, int tree_id) {
    int ans = 0;
    vector<change>::iterator a;
    for (i++; i; i -= i&(-i)){
      a = upper_bound(tree[i].begin(), tree[i].end(), change(0, tree_id)) - 1;
      ans = max(ans, a->data);
    }
    return ans;
  }
};
\end{lstlisting}

\newpage

\subsection{2D Sparse Table}

\cite{ShahjalalShohag2022Dec}

\hrulefill \vspace{-\baselineskip}
\begin{multicols}{2}
\inputminted[autogobble,fontsize=\tiny]{C++}{Data Structures/2d_sparse_table.cpp}
\end{multicols}
\vspace{-\baselineskip}
\noindent \hrulefill

\subsection{2D Sparse Segment Tree}

\hrulefill \vspace{-\baselineskip}
\begin{multicols}{2}
\inputminted[autogobble,fontsize=\tiny]{C++}{Data Structures/2d_sparse_segment_tree.cpp}
\end{multicols}
\vspace{-\baselineskip}
\noindent \hrulefill

\subsection{K-th Minimum in Segment Tree}

\newpage

\subsection{Treap}

\hrulefill \vspace{-\baselineskip}
\begin{multicols}{2}
\inputminted[autogobble,fontsize=\tiny]{C++}{Data Structures/treap.cpp}
\end{multicols}
\vspace{-\baselineskip}
\noindent \hrulefill

\newpage

\subsection{Min-Max Heap}

\hrulefill \vspace{-\baselineskip}
\begin{multicols}{2}
\inputminted[autogobble,fontsize=\tiny]{C++}{Data Structures/min_max_heap.cpp}
\end{multicols}
\vspace{-\baselineskip}
\noindent \hrulefill

\subsection{Indexed Binary Heap}

\hrulefill \vspace{-\baselineskip}
\begin{multicols}{2}
\inputminted[autogobble,fontsize=\tiny]{C++}{Data Structures/binary_indexed_heap.cpp}
\end{multicols}
\vspace{-\baselineskip}
\noindent \hrulefill

\subsection{Persistent Heap}

\hrulefill \vspace{-\baselineskip}
\begin{multicols}{2}
\inputminted[autogobble,fontsize=\tiny]{C++}{Data Structures/persistent_heap.cpp}
\end{multicols}
\vspace{-\baselineskip}
\noindent \hrulefill

\subsection{Skew Heap}

\hrulefill \vspace{-\baselineskip}
\begin{multicols}{2}
\inputminted[autogobble,fontsize=\tiny]{C++}{Data Structures/skew_heap.cpp}
\end{multicols}
\vspace{-\baselineskip}
\noindent \hrulefill

\newpage

\subsection{Persistent Array}

\hrulefill \vspace{-\baselineskip}
\begin{multicols}{2}
\inputminted[autogobble,fontsize=\tiny]{C++}{Data Structures/persistant_array.cpp}
\end{multicols}
\vspace{-\baselineskip}
\noindent \hrulefill

\subsection{Sparse Bitset}
\subsection{Interval Container}
\subsection{Stream Median}

TODO: Needs testing...

\begin{lstlisting}
template<typename T>
struct Median {
  priority_queue<T, vector<T>, greater<T>> right;
  priority_queue<T, vector<T>, less<T>> left;
  
  T get() {
    if (left.empty() && right.empty()) return -1;
    if (left.empty()) return right.top();
    if (right.empty()) return left.top();
    if (left.size() == right.size()) return left.top();
    return left.size() > right.size() ? left.top() : right.top();
  }

  void insert(T num) {
    T median = get();
    if (num < median) {
      if (left.size() > right.size()) {
        right.push(left.top());
        left.pop();
      }
      left.push(num);
    } else {
      if (right.size() > left.size()) {
        left.push(right.top());
        right.pop();
      }
      right.push(num);
    }
  }
};

\end{lstlisting}

\newpage

\subsection{Palindromic Trie}

\begin{center}
\begin{minipage}[t]{0.45\linewidth}
\begin{lstlisting}
struct palindromic_tree {
  vector<vector<int>> next;
  vector<int> suf, len;
  int new_node() {
    next.push_back(vector<int>(256,-1));
    suf.push_back(0);
    len.push_back(0);
    return next.size() - 1;
  }
  palindromic_tree(char *s) {
    len[new_node()] = -1;
    len[new_node()] =  0;
    int t = 1; 
    for (int i = 0; s[i]; ++i) {
      int p = t;
      for (; i-1-len[p] < 0 || 
             s[i-1-len[p]] != s[i];
             p = suf[p]);
      
      if ((t = next[p][s[i]]) >= 0) continue;
      t = new_node();
      len[t] = len[p] + 2;
      next[p][s[i]] = t;
      if (len[t] == 1) { 
        suf[t] = 1; // EMPTY
      } else {
        p = suf[p];
        for (; i-1-len[p] < 0 || s[i-1-len[p]] != s[i]; p = suf[p]);
        suf[t] = next[p][s[i]];
      }
    }
  }
\end{lstlisting}
\end{minipage}
\qquad
\begin{minipage}[t]{0.45\linewidth}
\begin{lstlisting}
  void display() {
    vector<char> buf;
    function<void (int, string)> rec =
      [&](int p, string depth) {
        if (len[p] > 0) {
          cout << depth;
          for (int i = buf.size()-1; i >= 0; --i)
            cout << buf[i];
          for (int i = len[p] % 2; i < buf.size(); ++i)
            cout << buf[i];
          cout << endl;
        }
        for (int a = 0; a < 256; ++a) {
          if (next[p][a] >= 0) {
            buf.push_back(a);
            rec(next[p][a], depth + " ");
            buf.pop_back();
          }
        }
      };
    
    cout << "---" << endl;
    rec(0, "");
    cout << "---" << endl;
    rec(1, "");
  }
};
\end{lstlisting}
\end{minipage}
\end{center}

\newpage

\subsection{Eval-Link-Update Tree}

\subsection{Euler Tour Tree}
The Euler Tour Tree is a dynamic connectivity based data structure \cite{Henzinger1998Nov} \cite{Miltersen1994Aug} capable of the following operations in $O(log(N))$ time:

{\setstretch{0.5}
\lstset{basicstyle={\small\ttfamily}\ttfamily,style=smaller_code}
\begin{itemize}
  \item{\lstinline{make_node(x)} ... return singleton with value x}
  \item{\lstinline{link(u,v)} ... add link between u and v}
  \item{\lstinline{cut(uv)} ... remove edge uv}
  \item{\lstinline{sum_in_component(u)} ... return sum of all values in the component}
\end{itemize}
}

Note that when adding a link between $u$ and $v$, they can't be already in the same component.
If the problem requires the sum of a component in a graph instead of a tree, we have explain
how you can achieve this using the Dynamic Connectivity.
\\
\\
The sum query can be interchanged with another query type
as long as the operation is in a associative field.


\begin{center}
\begin{minipage}[t]{0.45\linewidth}
\begin{lstlisting}
struct euler_tour_tree {
  struct node {
    int x, s; // value, sum
    node *ch[2], *p, *r;
  };
  int sum(node *t) { return t ? t->s : 0; }
  node *update(node *t) {
    if (t) t->s = t->x + sum(t->ch[0]) + sum(t->ch[1]);
    return t;
  }
  int dir(node *t) { return t != t->p->ch[0]; }
  void connect(node *p, node *t, int d) {
    p->ch[d] = t; if (t) t->p = p;
    update(p);
  }
  node *disconnect(node *t, int d) {
    node *c = t->ch[d]; t->ch[d] = 0; if (c) c->p = 0; 
    update(t);
    return c;
  }
  void rot(node *t) {
    node *p = t->p;
    int d = dir(t); 
    if (p->p) connect(p->p, t, dir(p));
    else      t->p = p->p;
    connect(p, t->ch[!d], d);
    connect(t, p, !d);
  }
  void splay(node *t) {
    for (; t->p; rot(t))
      if (t->p->p) rot(dir(t) == dir(t->p) ? t->p : t);
  }
  void join(node *s, node *t) {
    if (!s || !t) return;
    for (; s->ch[1]; s = s->ch[1]); splay(s);
    for (; t->ch[0]; t = t->ch[0]); connect(t, s, 0);
  }
\end{lstlisting}
\end{minipage}
\qquad
\begin{minipage}[t]{0.45\linewidth}
\begin{lstlisting}
  node *make_node(int x, node *l = 0, node *r = 0) {
    node *t = new node({x});
    connect(t, l, 0); connect(t, r, 1);
    return t;
  }
  node *link(node *u, node *v, int x = 0) {
    splay(u); splay(v); 
    node *uv = make_node(x, u, disconnect(v,1));
    node *vu = make_node(0, v, disconnect(u,1));
    uv->r = vu; vu->r = uv;
    join(uv, vu);
    return uv;
  }
  void cut(node *uv) {
    splay(uv); disconnect(uv,1); splay(uv->r); 
    join(disconnect(uv,0), disconnect(uv->r,1));
    delete uv, uv->r;
  }
  int sum_in_component(node *u) {
    splay(u);
    return u->s;
  }
};
\end{lstlisting}
\end{minipage}
\end{center}

\newpage

\subsection{Link-Cut Tree}

The Link-Cut Tree is one of the most powerful tree data structures.
It can support the following queries in amortized $O(log(n))$ time:

{\setstretch{0.5}
\lstset{basicstyle={\small\ttfamily}\ttfamily,style=smaller_code}
\begin{itemize}
  \item{\lstinline{link(u, v)} ... add an edge $(u, v)$}
  \item{\lstinline{cut(u)} ... cut the edge $(parent[u], u)$}
  \item{\lstinline{lca(u, v)} ... returns the least common ancestor of $u$ and $v$}
  \item{\lstinline{connected(u, v)} ... checks if $u$ and $v$ are part of the same tree}
  \item{\lstinline{find_root(u)} ... returns the root of $u$}
  \item{\lstinline{component_size(u)} ... returns the size of the component containing $u$}
  \item{\lstinline{subtree_size(u)} ... returns the subtree size of $u$}
  \item{\lstinline{depth(u)} ... returns the depth of $u$}
  \item{\lstinline{subtree_query(u, root)} ... subtree query with a given root}
  \item{\lstinline{query(u, v)} ... path from $u$ to $v$ - sum query}
  \item{\lstinline{update(u, x)} ... update the value of node $u$ to $x$}
  \item{\lstinline{update(u, v, x)} ... update the path from $u$ to $v$ with $x$}
\end{itemize}
\lstset{basicstyle=\fontsize{7}{9}\ttfamily,style=smaller_code}
}

The following implementation includes lazy propagation which allows updating of paths.
It is possible to change the sum function to any associative function.

\begin{center}
\begin{minipage}[t]{0.45\linewidth}
\begin{lstlisting}
struct node {
  int p = 0, c[2] = {0, 0}, pp = 0;
  bool flip = 0;
  int sz = 0, ssz = 0, vsz = 0;
  // sz  -> aux tree size
  // ssz -> subtree size in rep tree
  // vsz -> virtual tree size
  long long val = 0, sum = 0, lazy = 0;
  long long subsum = 0, vsum = 0;
  node() {}
  node(int x) {
    val = x; sum = x;
    sz = 1; lazy = 0;
    ssz = 1; vsz = 0;
    subsum = x; vsum = 0;
  }
};

struct LCT {
  vector<node> t;
  LCT(int n) : t(n + 1) {}

  // <independant splay tree code>
  int dir(int x, int y) { return t[x].c[1] == y; }
  void set(int x, int d, int y) {
    if (x) t[x].c[d] = y, pull(x);
    if (y) t[y].p = x;
  }
  void pull(int x) {
    if (!x) return;
    int &l = t[x].c[0], &r = t[x].c[1];
    t[x].sum = t[l].sum + t[r].sum + t[x].val;
    t[x].sz = t[l].sz + t[r].sz + 1;
    t[x].ssz = t[l].ssz + t[r].ssz + t[x].vsz + 1;
    t[x].subsum = t[l].subsum + t[r].subsum + t[x].vsum + t[x].val;
  }
  void push(int x) { 
    if (!x) return;
    int &l = t[x].c[0], &r = t[x].c[1];
    if (t[x].flip) {
      swap(l, r); 
      if (l) t[l].flip ^= 1; 
      if (r) t[r].flip ^= 1;
      t[x].flip = 0;
    }
    if (t[x].lazy) {
      t[x].val += t[x].lazy;
      t[x].sum += t[x].lazy * t[x].sz;
      t[x].subsum += t[x].lazy * t[x].ssz;
      t[x].vsum += t[x].lazy * t[x].vsz;
      if (l) t[l].lazy += t[x].lazy;
      if (r) t[r].lazy += t[x].lazy;
      t[x].lazy = 0;
    }
  }
  void rotate(int x, int d) { 
    int y = t[x].p, z = t[y].p, w = t[x].c[d];
    swap(t[x].pp, t[y].pp);
    set(y, !d, w);
    set(x, d, y);
    set(z, dir(z, y), x);
  }
  void splay(int x) { 
    for (push(x); t[x].p;) {
      int y = t[x].p, z = t[y].p;
      push(z); push(y); push(x);
      int dx = dir(y, x), dy = dir(z, y);
      if (!z) rotate(x, !dx); 
      else if (dx == dy) rotate(y, !dx), rotate(x, !dx); 
      else rotate(x, dy), rotate(x, dx);
    }
  }
  // </independant splay tree code>
\end{lstlisting}
\end{minipage}
\qquad
\begin{minipage}[t]{0.45\linewidth}
\begin{lstlisting}
  // making it a root in the rep. tree
  void make_root(int u) {
    access(u);
    int l = t[u].c[0];
    t[l].flip ^= 1;
    swap(t[l].p, t[l].pp);
    t[u].vsz += t[l].ssz;
    t[u].vsum += t[l].subsum;
    set(u, 0, 0);
  }
  // make the path from root to u a preferred path
  // returns last path-parent of a node as it moves up the tree
  int access(int _u) {
    int last = _u;
    for (int v = 0, u = _u; u; u = t[v = u].pp) {
      splay(u); splay(v);
      t[u].vsz -= t[v].ssz;
      t[u].vsum -= t[v].subsum;
      int r = t[u].c[1];
      t[u].vsz += t[r].ssz;
      t[u].vsum += t[r].subsum;
      t[v].pp = 0;
      swap(t[r].p, t[r].pp);
      set(u, 1, v);
      last = u;
    }
    splay(_u);
    return last;
  }
\end{lstlisting}
\end{minipage}
\end{center}
  
\begin{center}
\begin{minipage}[t]{0.45\linewidth}
\begin{lstlisting}
void link(int u, int v) { // u -> v
  // assert(!connected(u, v));
  make_root(v);
  access(u); splay(u);
  t[v].pp = u;
  t[u].vsz += t[v].ssz;
  t[u].vsum += t[v].subsum;
}
// cut par[u] -> u, u is non root vertex
void cut(int u) {
  access(u);
  assert(t[u].c[0] != 0);
  t[t[u].c[0]].p = 0;
  t[u].c[0] = 0;
  pull(u);
}
// parent of u in the rep. tree
int get_parent(int u) {
  access(u); splay(u); push(u);
  u = t[u].c[0]; push(u);
  while (t[u].c[1]) {
    u = t[u].c[1]; push(u);
  }
  splay(u);
  return u;
}
// root of the rep. tree containing this node
int find_root(int u) {
  access(u); splay(u); push(u);
  while (t[u].c[0]) {
    u = t[u].c[0]; push(u);
  }
  splay(u);
  return u;
}
bool connected(int u, int v) {
  return find_root(u) == find_root(v);
}
// depth in the rep. tree
int depth(int u) {
  access(u); splay(u);
  return t[u].sz;
}
int lca(int u, int v) {
  // assert(connected(u, v));
  if (u == v) return u;
  if (depth(u) > depth(v)) swap(u, v);
  access(v); 
  return access(u);
}
int is_root(int u) {
  return get_parent(u) == 0;
}
\end{lstlisting}
\end{minipage}
\qquad
\begin{minipage}[t]{0.45\linewidth}
\begin{lstlisting}
int component_size(int u) {
  return t[find_root(u)].ssz;
}
int subtree_size(int u) {
  int p = get_parent(u);
  if (p == 0) {
    return component_size(u);
  }
  cut(u);
  int ans = component_size(u);
  link(p, u);
  return ans;
}
long long component_sum(int u) {
  return t[find_root(u)].subsum;
}
long long subtree_sum(int u) {
  int p = get_parent(u);
  if (p == 0) {
    return component_sum(u);
  }
  cut(u);
  long long ans = component_sum(u);
  link(p, u);
  return ans;
}
// sum of the subtree of u when root is specified
long long subtree_query(int u, int root) {
  int cur = find_root(u);
  make_root(root);
  long long ans = subtree_sum(u);
  make_root(cur);
  return ans;
}
// path sum
long long query(int u, int v) {
  int cur = find_root(u);
  make_root(u); access(v);
  long long ans = t[v].sum;
  make_root(cur);
  return ans;
}
void update(int u, int x) {
  access(u); splay(u);
  t[u].val += x;
}
// add x to the nodes on the path from u to v
void update(int u, int v, int x) {
  int cur = find_root(u);
  make_root(u); access(v);
  t[v].lazy += x;
  make_root(cur);
}
\end{lstlisting}
\end{minipage}
\end{center}

\textbf{Example Problems}

\begin{problem}
  \textbf{- SPOJ - Dynamic Tree Connectivity} \cite{SPOJ_DYNACON1}
\end{problem}

\lstset{basicstyle={\small\ttfamily}\ttfamily,style=smaller_code}
\begin{problem}
  \textbf{- CODECHEF - Query on a tree VI} \cite{QTREE6}

  Given a tree, all the nodes are black initially. There are two types of queries:
\end{problem}
\begin{itemize}
  \item {\lstinline{Query 1}: How many nodes are connected to $u$, such that two nodes are connected iff all of the nodes on the path from $u$ to $v$ have the same color.}
  \item {\lstinline{Query 2}: Toggle the color of $u$.}
\end{itemize}
\lstset{basicstyle=\fontsize{7}{9}\ttfamily,style=smaller_code}

\subsection{Kinetic Tournament Data Structure}

\hrulefill \vspace{-\baselineskip}
\begin{multicols}{2}
\inputminted[autogobble,fontsize=\tiny]{C++}{Data Structures/kinetic.cpp}
\end{multicols}
\vspace{-\baselineskip}
\noindent \hrulefill