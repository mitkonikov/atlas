\subsection{2D Geometry}
\subsubsection{Helper Functions}
\subsubsection{Segment - Segment Intersection}
\subsubsection{Angle Struct}
\subsubsection{Center of Mass}
\subsubsection{Barycentric Coordinates}
\subsubsection{Point in Hull and Closest Pair of Points}
\subsubsection{Convex Hull}
\subsubsection{Online Convex Hull Merger}
\subsubsection{Convex Hull Container of Lines}
\subsubsection{Polygon Union}
\subsubsection{Minimum Circle that encloses all points}
\subsubsection{Convex Layers}

\newpage
\subsubsection{Check for segment pair intersection}

The following algorithm checks if any two of the $n$ segments intersect in $O(n \log{n})$ time.
Returns the indices of an intersecting pair.


\begin{center}
\begin{minipage}[t]{0.45\linewidth}
\begin{lstlisting}
const double EPS = 1E-9;

struct pt { double x, y; };

struct seg {
  pt p, q;
  int id;

  double get_y(double x) const {
    if (abs(p.x - q.x) < EPS) return p.y;
    return p.y + (q.y - p.y) * (x - p.x) / (q.x - p.x);
  }
};

bool intersect1d(double l1, double r1, double l2, double r2) {
  if (l1 > r1) swap(l1, r1);
  if (l2 > r2) swap(l2, r2);
  return max(l1, l2) <= min(r1, r2) + EPS;
}

int vec(const pt& a, const pt& b, const pt& c) {
  double s = (b.x - a.x) * (c.y - a.y) - (b.y - a.y) * (c.x - a.x);
  return abs(s) < EPS ? 0 : s > 0 ? +1 : -1;
}

bool intersect(const seg& a, const seg& b) {
  return intersect1d(a.p.x, a.q.x, b.p.x, b.q.x) &&
          intersect1d(a.p.y, a.q.y, b.p.y, b.q.y) &&
          vec(a.p, a.q, b.p) * vec(a.p, a.q, b.q) <= 0 &&
          vec(b.p, b.q, a.p) * vec(b.p, b.q, a.q) <= 0;
}

bool operator<(const seg& a, const seg& b) {
  double x = max(min(a.p.x, a.q.x), min(b.p.x, b.q.x));
  return a.get_y(x) < b.get_y(x) - EPS;
}

struct event {
  double x;
  int tp, id;

  event() {}
  event(double x, int tp, int id) : x(x), tp(tp), id(id) {}

  bool operator<(const event& e) const {
    if (abs(x - e.x) > EPS) return x < e.x;
    return tp > e.tp;
  }
};
\end{lstlisting}
\end{minipage}
\qquad
\begin{minipage}[t]{0.5\linewidth}
\begin{lstlisting}
set<seg> s;
vector<set<seg>::iterator> where;

set<seg>::iterator prev(set<seg>::iterator it) {
  return it == s.begin() ? s.end() : --it;
}

set<seg>::iterator next(set<seg>::iterator it) {
  return ++it;
}

pair<int, int> solve(const vector<seg>& a) {
  int n = (int)a.size();
  vector<event> e;
  for (int i = 0; i < n; ++i) {
    e.push_back(event(min(a[i].p.x, a[i].q.x), +1, i));
    e.push_back(event(max(a[i].p.x, a[i].q.x), -1, i));
  }
  sort(e.begin(), e.end());

  s.clear();
  where.resize(a.size());
  for (size_t i = 0; i < e.size(); ++i) {
    int id = e[i].id;
    if (e[i].tp == +1) {
      set<seg>::iterator nxt = s.lower_bound(a[id]), prv = prev(nxt);
      if (nxt != s.end() && intersect(*nxt, a[id]))
        return make_pair(nxt->id, id);
      if (prv != s.end() && intersect(*prv, a[id]))
        return make_pair(prv->id, id);
      where[id] = s.insert(nxt, a[id]);
    } else {
      set<seg>::iterator nxt = next(where[id]), prv = prev(where[id]);
      if (nxt != s.end() && prv != s.end() && intersect(*nxt, *prv))
        return make_pair(prv->id, nxt->id);
      s.erase(where[id]);
    }
  }

  return make_pair(-1, -1);
}
\end{lstlisting}
\end{minipage}
\end{center}
  
\subsubsection{Rectangle Union}
\subsubsection{Rotating Calipers}
\subsubsection{KD Tree}

\newpage
\subsubsection{Burkhard-Keller Tree}

Burkhard-Keller Tree \cite{Burkhard1973Apr} (also known as metric tree) is a flexible data structure
created to support the following queries:

{\setstretch{0.5}
\lstset{basicstyle={\small\ttfamily}\ttfamily,style=smaller_code}
\begin{itemize}
  \item{\lstinline{insert(p)} ... insert a point p in $O(log^2(n))$}
  \item{\lstinline{traverse(p, d)} ... enumerate all points q with $dist(p, q) \leq d$}
\end{itemize}
\lstset{basicstyle=\fontsize{7}{9}\ttfamily,style=smaller_code}
}

Note that the distance function in the following implementation uses the Chebyshev distance metric.
To change the distance metric, you need to redefine the distance function and redefine the check
inside the traverse function. To delete elements and/or rebalance the tree,
we can use the same technique as the scapegoat tree.

\begin{lstlisting}
typedef pair<int,int> PII;
int dist(PII a, PII b) { return max(abs(a.first - b.first), abs(a.second - b.second)); }
void process(PII a) { printf("%d %d\n", a.first, a.second); }

template <class T>
struct bk_tree {
  typedef int dist_type;
  struct node {
    T p;
    unordered_map<dist_type, node*> ch;
  } *root;
  bk_tree() : root(0) { }

  node *insert(node *n, T p) {
    if (!n) { n = new node(); n->p = p; return n; }
    dist_type d = dist(n->p, p);
    n->ch[d] = insert(n->ch[d], p);
    return n;
  }
  void traverse(node *n, T p, dist_type dmax) {
    if (!n) return;
    dist_type d = dist(n->p, p);
    if (d < dmax) {
      process(n->p); // write your process
    }
    for (auto i: n->ch) 
      if (-dmax <= i.first - d && i.first - d <= dmax) 
        traverse(i.second, p, dmax);
  }

  // Wrapper functions
  void insert(T p) { root = insert(root, p); }
  void traverse(T p, dist_type dmax) { traverse(root, p, dmax); }
};
\end{lstlisting}

\newpage

\subsubsection{Manhatten Minimum Spanning Tree}
\subsubsection{GJK Algorithm}
\subsubsection{Half-Plane intersection}
\subsection{3D Geometry}
\subsubsection{Point3D}
\subsubsection{3D Geometry}
\subsubsection{3D Convex Hull}
\subsubsection{Delaunay Triangulation}
\subsubsection{Voronoi Diagram with Euclidean Metric}
\subsubsection{Voronoi Diagram with Manhattan Metric}
\subsubsection{3D Coordinate-Wise Domination}
\subsubsection{Maximum Circle Cover}