\documentclass{article}


% NOTES:
% 
%   https://atcoder.jp/contests/abc222/editorial/2749
%   E Sum of Geometric Progression in O(log(N))
% 

% TO PRINT:
% Kirhov's Theorem
% Picks Theorem for Polygon area


\usepackage{listings}
\usepackage{color}
\usepackage{multicol}
\usepackage{hyperref}
\usepackage{setspace}
\usepackage{amssymb}
\usepackage[margin=1in]{geometry}
\usepackage{import}
\usepackage{minted}

\hypersetup{
    colorlinks,
    citecolor=black,
    filecolor=black,
    linkcolor=black,
    urlcolor=black
}

\definecolor{dkgreen}{rgb}{0,0.6,0}
\definecolor{gray}{rgb}{0.5,0.5,0.5}
\definecolor{mauve}{rgb}{0.58,0,0.82}

\lstset{frame=tb,
  language=c++,
  aboveskip=3mm,
  belowskip=3mm,
  showstringspaces=false,
  columns=flexible,
  basicstyle={\small\ttfamily},
  numbers=none,
  numberstyle=\tiny\color{gray},
  keywordstyle=\color{blue},
  commentstyle=\color{dkgreen},
  stringstyle=\color{mauve},
  breaklines=true,
  breakatwhitespace=true,
  tabsize=3
}

\lstdefinestyle{smaller_code}{
  language=c++,
}

\title{\textbf{Competitive Programming Library}}
% \author{	\vspace{20pt}Mitko Nikov}
\date{\today}
\author{}

\newtheorem{theorem}{Theorem}
\newtheorem{lemma}{Lemma}
\newtheorem{proof}{Proof}
\newtheorem{remark}{Remark}
\newtheorem{problem}{Problem}
\begin{document}

\maketitle

\newpage

\vspace*{\fill}
\begin{center}
  Intentionally left blank.
\end{center}
\vspace*{\fill}

\newpage

\tableofcontents

\newpage

\section{Foreword}

Dear Reader, I collected this set of algorithms and data structures from various sources
over the years and now is the time to give it all to the world. This collection of algorithms
and data structures contains some very well known ones, but also some that are so specific, that
they have a single use case only.

Some of the algorithms, for shorter code use the following template:

\begin{lstlisting}
#define rep(i, a, b) for(int i = a; i < (b); ++i)
#define all(c) ((c).begin()), ((c).end())
#define sz(x) (int)(x).size()
typedef long long ll;
typedef pair<int, int> pii;
typedef vector<int> vi;
\end{lstlisting}

Always try to do the following checks before submitting a problem:
\begin{itemize}
  \item Edge cases?
  \item Overflows?
  \item Memory allocation (MLE)?
  \item Out-of-bounds on array access?
  \item Did you escape all characters?
  \item Recursion depth, stack memory?
\end{itemize}

\noindent \hrulefill

\textbf{I dedicate this book to You - all of the people that work hard, even if they know
that they will never be appreciated for what they have accomplished or what they tried to achieve.}

\section{Graphs}

\import{./Graphs}{index.tex}

\newpage

\section{Trees}

\import{./Trees}{index.tex}

\newpage

\section{Algorithms}

\import{./Algorithms}{index.tex}

\newpage

\section{String Algorithms}

\import{./Strings}{index.tex}

\newpage

\section{Data Structures}

\import{./Data Structures}{index.tex}

\newpage

\section{Maths}

\import{./Maths}{index.tex}

\newpage

\section{Hashing}

\import{./Hashing}{index.tex}

\newpage

\section{Geometry}

\import{./Geometry}{index.tex}

\newpage

\section{Miscellaneous Stuff}

\import{./Misc}{index.tex}

\newpage

\section{Picked Solutions}

\import{./Solutions}{index.tex}

\newpage
\section{References}
\bibliographystyle{plain}
\bibliography{refs}

\end{document}